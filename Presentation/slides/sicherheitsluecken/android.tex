\begin{frame}
	\textbf{Exynos-Exploit} (2012)
	\begin{block}{}
		Sicherheitslücke im Kernel von Android Geräten von Samsung
	\end{block}		
	\begin{block}{}
		/dev/exynos-mem ist für \textbf{alle} Nutzer nutzbar!
	\end{block}
	\begin{block}{}
		Enthält den kompletten physikalischen RAM
	\end{block}
	\begin{block}{}
		Folgen: RAM Dump, Code Injection, Rooting, etc. möglich
	\end{block}
\end{frame}
\begin{frame}
	\textbf{Stagefright} (2015)
	\begin{block}{}
		Sicherheitslücken in mehreren Bibliotheken (libstagefright, libutils)
	\end{block}		
	\begin{block}{}
		Betrifft ca. 95\% aller Android Geräte
	\end{block}
	\begin{block}{}
		Manipulierte mp3s und mp4s
	\end{block}
	\begin{block}{}
		Remote Code Execution mit Rechten der Library möglich						
	\end{block}
	\begin{block}{}
		Erste Patches wurden ausgeliefert
	\end{block}
\end{frame}