\documentclass[aspectratio=169]{beamer}

\mode<presentation>
{
\usetheme{Darmstadt}%Darmstadt,Frankfurt

\setbeamercovered{transparent}
}
%Deutsche Silbentrennung
\usepackage[ngerman]{babel}
%Deutsche Umlaute
\usepackage[utf8]{inputenc}
%Listen einrücken
\usepackage{enumitem}
% font definitions, try \usepackage{ae} instead of the following
% three lines if you don't like this look
\usepackage{mathptmx}
\usepackage[scaled=.90]{helvet}
\usepackage{courier}
%Trennung von deutschen Umlauten
\usepackage[T1]{fontenc}
\usepackage{adjustbox}
\usepackage{url}

\title{Sicherheit in Android und iOS}

%\subtitle{}

% - Use the \inst{?} command only if the authors have different
%   affiliation.
%\author{F.~Author\inst{1} \and S.~Another\inst{2}}
\author{David Artmann \and Kristoffer Schneider}

% - Use the \inst command only if there are several affiliations.
\institute[Universities of]
{
%\inst{1}
\inst{}
Hochschule für angewandte Wissenschaften\\
Würzburg-Schweinfurt
}

\date{\today}


% This is only inserted into the PDF information catalog. Can be left
% out.
\subject{Talks}



% If you have a file called "university-logo-filename.xxx", where xxx
% is a graphic format that can be processed by latex or pdflatex,
% resp., then you can add a logo as follows:
\pgfdeclareimage[height=0.5cm]{university-logo}{media/logo/fhws.png}
\logo{\pgfuseimage{university-logo}}



% Delete this, if you do not want the table of contents to pop up at
% the beginning of each subsection:
\AtBeginSubsection[]
{
\begin{frame}<beamer>
\frametitle{Gliederung}
\tableofcontents[currentsection,currentsubsection]
\end{frame}
}

% If you wish to uncover everything in a step-wise fashion, uncomment
% the following command:
\beamerdefaultoverlayspecification{<+->}

\begin{document}
% titlepage
\input{slides/general/titlepage.tex}
% toc
\begin{frame}
	\frametitle{Gliederung}
	\tableofcontents
	% You might wish to add the option [pausesections]
\end{frame}

% Section Gemeinsamkeiten und Unterschiede
\section[Systemsicherheit]{Systemsicherheit}
	\begin{frame}
%\frametitle{Gemeinsamkeiten und Unterschiede}
%\framesubtitle{Systemsicherheit}
\centering
	\begin{columns}[T] % contents are top vertically aligned
    	\begin{column}[T]{5cm} % each column can also be its own environment
    		\center{\includegraphics[height=0.2\linewidth]{media/graphics/Android_robot.png}}
    		\begin{block}{}
				Trusted Execution Environment
			\end{block}
    	\end{column}
    	\begin{column}[T]{5cm}
    		\center{\includegraphics[height=0.2\linewidth]{media/graphics/Apple_logo_black.png}}
    		\begin{block}{}
    			Secure Enclave
    		\end{block}
    	\end{column}
    \end{columns}
    
    \begin{columns}[c]
    	\begin{column}[c]{12cm}
		    \begin{block}{}
				\center{Secure boot chain}
			\end{block}
		\end{column}
	\end{columns}
	
    \begin{columns}[t]
    	\begin{column}[t]{5cm}
    		\begin{block}{}
				Pro App ein Linux-Nutzer
			\end{block}
			\begin{block}{}
				Isolation durch Kernel
			\end{block}
    	\end{column}
    	\begin{column}[t]{5cm}
			\begin{block}{}
				Alle Apps ein Unix-Nutzer
			\end{block}
			\begin{block}{}
				Sandbox pro App definiert
			\end{block}
    	\end{column}
    \end{columns}
\end{frame}
	% Subsection Systemsicherheit
	%\subsection[Systemsicherheit]{Systemsicherheit}
		%\begin{frame}
%\frametitle{Gemeinsamkeiten und Unterschiede}
%\framesubtitle{Systemsicherheit}
\centering
	\begin{columns}[T] % contents are top vertically aligned
    	\begin{column}[T]{5cm} % each column can also be its own environment
    		\center{\includegraphics[height=0.2\linewidth]{media/graphics/Android_robot.png}}
    		\begin{block}{}
				Trusted Execution Environment
			\end{block}
    	\end{column}
    	\begin{column}[T]{5cm}
    		\center{\includegraphics[height=0.2\linewidth]{media/graphics/Apple_logo_black.png}}
    		\begin{block}{}
    			Secure Enclave
    		\end{block}
    	\end{column}
    \end{columns}
    
    \begin{columns}[c]
    	\begin{column}[c]{12cm}
		    \begin{block}{}
				\center{Secure boot chain}
			\end{block}
		\end{column}
	\end{columns}
	
    \begin{columns}[t]
    	\begin{column}[t]{5cm}
    		\begin{block}{}
				Pro App ein Linux-Nutzer
			\end{block}
			\begin{block}{}
				Isolation durch Kernel
			\end{block}
    	\end{column}
    	\begin{column}[t]{5cm}
			\begin{block}{}
				Alle Apps ein Unix-Nutzer
			\end{block}
			\begin{block}{}
				Sandbox pro App definiert
			\end{block}
    	\end{column}
    \end{columns}
\end{frame}
\section[Applikationssicherheit]{Applikationssicherheit}
	\subsection{Berechtigungen}
		\begin{frame}
	\center{\includegraphics[height=0.08\linewidth]{media/graphics/Android-ios-clone.png}}
	\begin{block}{}
		iOS bis Android M granularer
	\end{block}
	\begin{block}{}
		Abhilfe durch AppOps (<=4.4.2) oder XPrivacy (>=4.0.3 \&\& <=5.1.1)
	\end{block}
	\begin{block}{}
		Mit iOS 9 und Android M gleichauf
	\end{block}
\end{frame}
		\begin{frame}
	\centering
	\includegraphics[height=0.5\linewidth]{media/graphics/permissions}
\end{frame}
	\subsection{Distribution}
		\begin{frame}
	\center{\includegraphics[height=0.08\linewidth]{media/graphics/Android-ios-clone.png}}
	
	\begin{block}{}
		\visible<1->{
			iOS nur über Apple's App Store
		}
		\visible<2->{
			\begin{tabular}{c}
				\includegraphics[height=0.5cm]{media/graphics/apple-app-store}
			\end{tabular}
		}
	\end{block}	
	
	\visible<3->{
		\begin{block}{}
				Android bietet diverse
			\visible<4->{
				\begin{tabular}{c}
					\includegraphics[height=0.5cm]{media/graphics/google-play}
					
				\end{tabular}
			}
			\visible<5->{
				\begin{tabular}{c}
					\includegraphics[height=0.5cm]{media/graphics/fdroid}
				\end{tabular}
			}
			\visible<6->{
				\begin{tabular}{c}
					\includegraphics[height=0.5cm]{media/graphics/amazon-app-shop}
				\end{tabular}
			}
		\end{block}
	}
\end{frame}
	% Subsection Applikationssicherheit
	%\subsection[Applikationssicherheit]{Applikationssicherheit}	
\section[Aktuelle Sicherheitslücken]{Aktuelle Sicherheitslücken}
	\subsection{Android}
		\begin{frame}
	\textbf{Exynos-Exploit} (2012)
	\begin{block}{}
		Sicherheitslücke im Kernel von Android Geräten von Samsung
	\end{block}		
	\begin{block}{}
		/dev/exynos-mem ist für \textbf{alle} Nutzer nutzbar!
	\end{block}
	\begin{block}{}
		Enthält den kompletten physikalischen RAM
	\end{block}
	\begin{block}{}
		Folgen: RAM Dump, Code Injection, Rooting, etc. möglich
	\end{block}
\end{frame}
\begin{frame}
	\textbf{Stagefright} (2015)
	\begin{block}{}
		Sicherheitslücken in mehreren Bibliotheken (libstagefright, libutils)
	\end{block}		
	\begin{block}{}
		Betrifft ca. 95\% aller Android Geräte
	\end{block}
	\begin{block}{}
		Manipulierte mp3s und mp4s
	\end{block}
	\begin{block}{}
		Remote Code Execution mit Rechten der Library möglich						
	\end{block}
	\begin{block}{}
		Erste Patches wurden ausgeliefert
	\end{block}
\end{frame}
	\subsection{iOS}
		%\begin{frame}
	\centering
	\textbf{No iOS Zone} (2014)
	\begin{block}{}
		\url{https://www.youtube.com/watch?v=i2tYdmOQisA}
	\end{block}
	\begin{block}{}
		Verbinden zu WLAN-AP führt zu DoS und Bootloop
	\end{block}
	\begin{block}{}
		Fehler im Parser für SSL-Zertifikate
	\end{block}
	\begin{block}{}
		Behoben mit iOS 8.3
	\end{block}
\end{frame}
		\begin{frame}
	\centering
	\textbf{XcodeGhost} (Q3 2015)
	\begin{block}{}
		Compiler der Xcode IDE überprüft keine externen Bibliotheken
	\end{block}
	\begin{block}{}
		Sources aus inoffiziellen Kanälen (FW der Regierung)
	\end{block}
	\begin{block}{}
		Daten wurden an C\&C-Server des Autors versandt
	\end{block}
	\begin{block}{}
		GateKeeper würde durch Codesignaturprüfung ausführen von Xcode verhindern
	\end{block}
	\begin{block}{}
		Sandbox weiterhin aktiv (legitimes Verhalten!)
	\end{block}
\end{frame}
		%\begin{frame}
	\centering
	\textbf{AirDrop Exploit} (Q3 2015)
	\begin{block}{}
		Schadcodeverteilung über AirDrop (iOS 7 - 8.4.1, OS X >= Yosemite)
	\end{block}
	\begin{block}{}
		Über directory traversal Angriff wird Payload auch bei Ablehnung der Daten
		geschrieben
	\end{block}
	\begin{block}{}
		Apps über Developer Enterprise Program signiert -> kein AppStore!
	\end{block}
	\begin{block}{}
		"`Trust-prompt"' lässt sich durch enterprise provisioning profile unterdrücken
	\end{block}
\end{frame}
\section{Härten}
	\subsection{Ratschläge für Entwickler}
		\begin{frame}
	\centering
	\textbf{Ratschläge für Entwickler}
	\center{\includegraphics[height=0.1\linewidth]{media/graphics/Android-ios-clone.png}}
	\begin{block}{}
		Passwortrichtlinie
	\end{block}
	\begin{block}{}
		Sicherung des Hauptschlüssels
	\end{block}
	\begin{block}{}
		Lokations- und Tempusberücksichtigung
	\end{block}
	\begin{block}{}
		Bipartite Schlüssel
	\end{block}
	\begin{block}{}
		Manipulationsschutz
	\end{block}
\end{frame}
		%\begin{frame}
	\centering
	\textbf{Ratschläge für Entwickler}
	
	\begin{columns}[T] % contents are top vertically aligned
    	
    	\begin{column}[T]{5cm} % each column can also be its own environment
    		\center{\includegraphics[height=0.3\linewidth]{media/graphics/Android_robot.png}}
    		\begin{block}{}
				Auf Eigenschaften der Zielversion achten
			\end{block}
			\begin{block}{}
				Auf neue Permissions achten
			\end{block}
    	\end{column}
    	
    	\begin{column}[T]{5cm} % each column can also be its own environment
    		\center{\includegraphics[height=0.3\linewidth]{media/graphics/Apple_logo_black.png}}
    		\begin{block}{}
				Common Crypto Library
			\end{block}
			\begin{block}{}
				Jailbreakerkennung
			\end{block}
    	\end{column}	
    	
    \end{columns}
\end{frame}
	\subsection{Tips für Endnutzer}
		\begin{frame}
	\centering
	\textbf{Tips für Endnutzer}
	\center{\includegraphics[height=0.1\linewidth]{media/graphics/Android-ios-clone.png}}
	\begin{block}{}
		Lockscreen nutzen und konfigurieren
	\end{block}
	\begin{block}{}
		Zwei-Faktor-Authentifizierung
	\end{block}
	\begin{block}{}
		Privacy Einstellungen
	\end{block}
\end{frame}
\end{document}
